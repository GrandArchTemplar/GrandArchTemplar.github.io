\documentclass{paper}

% Language setting
% Replace `english' with e.g. `spanish' to change the document language
\topmargin=-20mm
\textheight=24cm
\textwidth=19cm
\oddsidemargin=-20mm % отступ от левого края нечетные страницы
\evensidemargin=-20mm %

\usepackage[T2A]{fontenc}
\usepackage[utf8]{inputenc}
\usepackage[russian]{babel}
\usepackage{amsmath}
\usepackage{amssymb}
\usepackage{graphicx}

\begin{document}
\section{Внимание!}
Для оценки 5 необходимо не только сделать задания, но и сдать экзамен
\section{Теория множеств}
\begin{enumerate}
    \item Из цифр ису составьте множество А, 
    из букв имени составить множество В, 
    из букв фамилии составить множество С. Найти $A \cup (B \Delta C)$
    \item Найти $A \times B$, где $A = \{1,2,3\}, B = \{2,3\}$
    \item Дано множество $U = \{1,2,3,4,5,6,7,8\}$. $A$ множество четных чисел, $B$ множество чисел меньше 5, $C$ множество нечетных чисел.
    Найти $2^A - 2^{(B \cap C)}$
    \item Найти свойства отношения(транзитивность, симметричность, рефлексивность) 
    $aRb: ab + a + b + 1$ кратно 6
    \item Из 300 человек, 144 пьют кофе по утрам, 126 едят яичницу по утрам и 111 читают газету по утрам.
    При этом яичницу или кофе по утрам на завтрак имеют 228 человек. 
    Кофе или газету по утрам имеют 228 человек, а яичницу или газету 198 человек.
    Все три вещи на завтрак имеют 15 человек. 
    Сколько человек не пьют кофе и не едят яичницу и не читают газеты по утрам?
    \item Решить уравнение относительно $X$: $AX \cup B = A - X$
\end{enumerate}
\section{Булева алгебра}
\begin{enumerate}
    \item Построить таблицу истинности функции $w \land (x \rightarrow (y \lor z))$
    \item Построить СКНФ функции $y \land (x \rightarrow (y \lor z))$
    \item Построить полином Жегалкина $y \land (\lnot x \rightarrow (\lnot x \land z))$
    \item Определить классы Поста функции $y \lor (x \rightarrow (y \land \lnot z))$
    \item Найти все бинарные булевы функции не являющиеся симметричными и являющиеся линейными
    \item Опровергнуть или доказать полноту набора $\{\rightarrow, \otimes\}$
\end{enumerate}
\section{Комбинаторика}
\begin{enumerate}
    \item Сколько существует трехзначных чисел кратных 5?
    \item В группе 22 человека.
    Стипендию могут получить только трое.
    Сколько вариантов различных конфигураций стипендиатов существует?
    \item 20 гонщиков участвовали в заездах. 
    Сколько конфигураций призовых мест (первые три места) существует?
    \item Сколько из цифр 1, 5, 8, 9 (без повторения) можно составить четырехзначных чисел, у которых вторая цифра 8?
    \item Чему равен коэффициент при $x^4y^8$ в разложении $(1 + x + y)^{20}$
    \item Сколько натуральных чисел от 1 до 1000 
    не делится ни на 2, 
    ни на 14, 
    но при этом делится на 7 и делится на 3 или 5, 
    но не на 15
\end{enumerate}
\section{Теория графов}
\begin{enumerate}
    \item Найти радиус, диаметр и центр графа заданного матрицей смежности:
    \[
        \begin{pmatrix}
            0 & 1 & 0 & 1 & 0 & 0 & 0 & 1\\
            1 & 0 & 1 & 0 & 0 & 0 & 1 & 1\\
            0 & 1 & 0 & 1 & 0 & 1 & 0 & 0\\
            1 & 0 & 1 & 0 & 1 & 0 & 0 & 0\\
            0 & 0 & 0 & 1 & 0 & 0 & 1 & 0\\
            0 & 0 & 1 & 0 & 0 & 0 & 1 & 0\\
            0 & 1 & 0 & 0 & 1 & 1 & 0 & 1\\
            1 & 1 & 0 & 0 & 0 & 0 & 1 & 0\\
        \end{pmatrix}
    \]
    \item Построить граф, центр которого состоит из трех вершин и не совпадает с множеством всех вершин
    \item Найти граф с числом вершин больше чем 1 такой что граф и его дополнение связаны
    \item Найти граф с шестью вершинами, который имеет эйлеров цикл, но не имеет гамильтонова цикла
    \item Сколько ребер в связном графе с $n$ вершинами, если в нем имеется единственный цикл? 
    \item При каких $n$ $n$-мерный куб планарен?
\end{enumerate}
\section{Темы для экзамена}
\begin{enumerate}
    \item Операции над множествами
    \item Бинарные отношения
    \item Булевы функции
    \item Нормальные формы (СКНФ, СДНФ, Полином Жегалкина)
    \item Классы Поста
    \item Теорема Поста
    \item Бином Ньютона
    \item Сочетания, размещения, перестановки
    \item Представления графов
    \item Классификация графов
    \item Характеристики графов
    \item Обходы графов
    \item Эйлеровы и гамильтоновы пути
\end{enumerate}
\end{document}